% !TEX TS-program = pdflatexmk

\documentclass[11pt]{article}
\usepackage{syllable-struc-forest}
\def\syllableversion{1.2}
\def\syllabledate{2018/09/16}
\title{\textbf{Using the  syllable-struc-forest package}}
\author{\textbf{Alan Munn}\\Department of Linguistics and Languages\\\texttt{\href{mailto:amunn@msu.edu}{amunn@msu.edu}}}
\date{Version \syllableversion\\\syllabledate}
\usepackage[T1]{fontenc}
\usepackage{parskip}
\usepackage[lmargin=.75in,rmargin=.75in,tmargin=1in,bmargin=1in]{geometry}
\usepackage{titling}
\usepackage{array, booktabs, multicol,tabularx, fancyhdr,xspace}
\usepackage{enumitem}
\usepackage{fancyvrb,listings,url}
\usepackage[sf]{titlesec}
\usepackage[colorlinks=true]{hyperref}
\usepackage{capt-of}
\usepackage{gb4e}



\DefineShortVerb{\|}
\newcommand*\bs{\textbackslash}


\IfFileExists{luximono.sty}%
{%
  \usepackage[scaled]{luximono}%
}
{%
  \IfFileExists{beramono.sty}%
  {%
    \usepackage[scaled]{beramono}%
  }{}
}

  
\lstset{%
    basicstyle=\ttfamily\small,
    commentstyle=\itshape\ttfamily\small,
    showspaces=false,
    showstringspaces=false,
    breaklines=true,
    breakautoindent=true,
    frame=single
    captionpos=t
    language=TeX
}
  
\newcommand*{\pkg}[1]{\texttt{#1}\xspace}

\setlength{\droptitle}{-1in}

\lhead{}
\chead{}
\rhead{}
\lfoot{\emph{}}
\cfoot{\thepage}
\rfoot{}
\renewcommand{\headrulewidth}{0pt}
\renewcommand{\footrulewidth}{0pt}
\pagestyle{fancy}


\begin{document}
\maketitle
\thispagestyle{empty}
\renewcommand{\abstractname}{\sffamily Abstract}
\begin{abstract}\noindent\begin{quote} This is a package that allows you to draw syllable trees quickly with a simple input.\end{quote}
\end{abstract}
\section{Package options}The package has one option: |[xslot]| which displays timing units as $\times$. If this option is not specified, segments are dominated simply by C or V as appropriate.  The choice can be  selectively changed within a document using the macros |\xslottrue| and |\xslotfalse|.


\section{Package commands}
\subsection{Syllable commands}
The package provides commands for the  range of syllables shown in Table~\ref{syllables}:  The commands are straightforward: each macro consists simply of the CV specification of the syllable, and takes as many arguments as slots in that syllables.  Each macro takes an optional argument to pass extra \pkg{forest} options if necessary. For most uses this will not be needed.

\begin{center}
\captionof{table}{Package syllable commands}\label{syllables}
\medskip
\begin{tabular}{lc}
\toprule
\multicolumn{1}{c}{Example Command} & Number of arguments \\
\midrule
|\V{a}| & 1\\
|\CV{b}{a}| & 2\\
|\VC{a}{b}| & 2\\
|\CVC{k}{a}{v}| & 3\\
|\CCV{s}{t}{a}| & 3\\
|\VCC{a}{l}{p}| & 3\\
|\CCVC{s}{t}{a}{l}| & 4\\
|\CVCC{s}{a}{l}{t}| & 4\\
|\CCCV{s}{t}{r}{i}| & 4\\
|\CCVCC{s}{l}{a}{r}{p}| & 5\\
|\CCCVC{s}{t}{r}{i}{k}| & 5\\
|\CCCVCC{s}{t}{r}{i}{l}{k}| & 6\\
|\CCCVCCC{s}{t}{r}{i}{l}{k}{s}| & 7\\
\bottomrule
\end{tabular}
\end{center}
\section{Long segments}
All of the macros listed in Table~{\ref{syllables}} also have versions in which the V slot is doubled. If the two vowels provided as arguments are the same, then the single vowel will be doubly linked to two V slots. These macros therefore have one more argument each). For example: |\CVVC{b}{a}{a}{b}| will yield a doubly linked \emph{a} vowel, as in (\ref{long}a)  while |\CVVC{b}{a}{i}{b}| will link each vowel to its own slot as in (\ref{long}b):

\begin{multicols}{2}
\begin{exe}
\ex\label{long}
\begin{xlist}
\ex \CVVC[baseline]{b}{a}{a}{b}
\ex \CVVC[baseline]{b}{a}{i}{b}
\end{xlist}
\end{exe}
\end{multicols}

The same is true for any CC cluster (onset or coda). If the consonant in the two slots is identical, they will be linked to a single consonant. (This does not apply to CC clusters that are part of CCC clusters.)

The width of the double linking (for both vowels and consonants) is controlled by the length |\longVlen|. This is set  initially to |-5pt|.

\section{IPA Fonts}
When used with pdfLaTeX, the \pkg{tipa} package is automatically loaded, and segments are rendered in TIPA automatically.

When used with XeLaTeX or LuaLaTeX, Linux Libertine O is loaded as the default IPA font. If you are using another font for your IPA symbols, define the font using \pkg{fontspec}.  For example if you are using Doulos SIL as your IPA font, you would use  |\newfontfamily\myipafont{Doulos SIL}|) and then use |\setIPAfont{\myipafont}| to set the font used in the syllables.

\section{Other user macros}
\begin{center}
\captionof{table}{Other user macros}\label{macros}
\medskip
\begin{tabularx}{.8\linewidth}{>{\ttfamily}lX}
\toprule
\multicolumn{1}{c}{Macro} & Description \\
\midrule
\bs setIPAfont\{<fontmacro>\} & for XeLaTeX or LuaLaTeX: set the IPA font to \texttt{<fontmacro>}\\
\bs xslottrue & render subsequent slots as $\times$\\
\bs xslotfalse & render subsequent slots as C or V\\
\bs longVlen & spacing adjustment for long vowels. \\
\bottomrule
\end{tabularx}
\end{center}
\clearpage
\section{Examples}
\begin{multicols}{3}
|\V{a}|

\V{a}

|\VV{a}{a}|

\VV{a}{a}

|\VV{a}{i}|

\VV{a}{i}
\end{multicols}
\begin{multicols}{3}
|\CV{b}{a}|

\CV{b}{a}

|\CVV{b}{a}{a}|

\CVV{b}{a}{a}

|\CVV{b}{a}{i}|

\CVV{b}{a}{i}

\end{multicols}
\xslottrue
\begin{multicols}{3}
|\VC{a}{b}|

\VC{a}{b}

|\VVC{a}{a}{b}|

\VVC{a}{a}{b}

|\VVC{a}{i}{b}|

\VVC{a}{i}{b}
\end{multicols}
\clearpage
\begin{multicols}{3}
|\CVC{k}{a}{v}|

\CVC{k}{a}{v}

|\CVVC{k}{a}{a}{b}|

\CVVC{k}{a}{a}{b}

|\CVVC{k}{a}{i}{b}|

\CVVC{k}{a}{i}{b}
\end{multicols}
%\clearpage
\begin{multicols}{3}
|\CCV{s}{t}{a}|

\CCV{s}{t}{a}

|\CCVV{s}{t}{a}{a}|

\CCVV{s}{t}{a}{a}

|\CCVV{s}{t}{a}{i}|

\CCVV{s}{t}{a}{i}
\end{multicols}
\begin{multicols}{3}
|\VCC{a}{l}{p}|

\VCC{a}{l}{p}

|\VVCC{a}{a}{l}{p}|

\VVCC{a}{a}{l}{p}

|\VVCC{a}{i}{l}{p}|

\VVCC{a}{i}{l}{p}
\end{multicols}
\clearpage
\xslotfalse
\begin{multicols}{3}
|\CCVC{s}{t}{a}{l}|

\CCVC{s}{t}{a}{l}

|\CCVVC{s}{t}{a}{a}{l}|

\CCVVC{s}{t}{a}{a}{l}

|\CCVVC{s}{t}{a}{i}{l}|

\CCVVC{s}{t}{a}{i}{l}
\end{multicols}
%\clearpage
\begin{multicols}{3}
|\CVCC{s}{a}{l}{t}|

\CVCC{s}{a}{l}{t}

|\CVVCC{s}{a}{a}{l}{t}|

\CVVCC{s}{a}{a}{l}{t}

|\CVVCC{s}{a}{i}{l}{t}|

\CVVCC{s}{a}{i}{l}{t}
\end{multicols}
\begin{multicols}{3}
|\CCVCC{s}{l}{a}{r}{p}|

\CCVCC{s}{l}{a}{r}{p}

|\CCVVCC{s}{l}{a}{a}{r}{p}|

\CCVVCC{s}{l}{a}{a}{r}{p}

|\CCVVCC{s}{l}{a}{i}{r}{p}|

\CCVVCC{s}{l}{a}{i}{r}{p}
\end{multicols}
\clearpage
\xslottrue
\begin{multicols}{3}
|\CCCV{s}{t}{r}{i}|

\CCCV{s}{t}{r}{i}

|\CCCVV{s}{t}{r}{a}{a}|

\CCCVV{s}{t}{r}{a}{a}

|\CCCVV{s}{t}{r}{a}{i}|

\CCCVV{s}{t}{r}{a}{i}
\end{multicols}
%\clearpage
\begin{multicols}{3}

|\CCCVC{s}{t}{r}{i}{k}|

\CCCVC{s}{t}{r}{i}{k}

|\CCCVVC{s}{t}{r}{a}{a}{k}|

\CCCVVC{s}{t}{r}{a}{a}{k}

|\CCCVVC{s}{t}{r}{a}{i}{k}|

\CCCVVC{s}{t}{r}{a}{i}{k}
\end{multicols}
\begin{multicols}{2}
|\CCCVCC{s}{t}{r}{i}{l}{k}|

\CCCVCC{s}{t}{r}{i}{l}{k}

|\CCCVVCC{s}{t}{r}{a}{a}{l}{k}|

\CCCVVCC{s}{t}{r}{a}{a}{l}{k}
\end{multicols}
\begin{multicols}{2}
|\CCCVVCC{s}{t}{r}{a}{i}{l}{k}|

\CCCVVCC{s}{t}{r}{a}{i}{l}{k}

|\CCCVCCC{s}{t}{r}{i}{l}{k}{s}|

\CCCVCCC{s}{t}{r}{i}{l}{k}{s}
\end{multicols}
%\clearpage
\xslotfalse
\begin{multicols}{2}

|\CCCVVCCC{s}{t}{r}{a}{a}{l}{k}{s}|

\CCCVVCCC{s}{t}{r}{a}{a}{l}{k}{s}

|\CCCVVCCC{s}{t}{r}{a}{i}{l}{k}{s}|

\CCCVVCCC{s}{t}{r}{a}{i}{l}{k}{s}
\end{multicols}



\end{document}
